\documentclass[12pt]{article}
\usepackage{amsmath}
\usepackage{graphicx}
\usepackage{float}
\begin{document}
\title{Electrical Engineering 141, Homework 6}
\date{March 1st, 2019}
\author{Michael Wu\\UID: 404751542}
\maketitle

\section*{Problem 1}

The open loop transfer function is the following.
\[\frac{K}{(s+1)^3}\]
This crosses the real axis when \(1+j\omega\) has a phase of \(\pm \frac{\pi}{3}\) or \(0\). This occurs
at \(\omega=0,\pm\sqrt{3}\). The magnitude of the open loop transfer function at these points are
\(K\) and \(-\frac{K}{8}\), respectively. The open loop transfer function has no poles in the right half
plane, so we should have no encirclements around \(-1\) in order to have a stable system. Since we know
that the value of the transfer function goes to zero as \(\omega\) goes to \(\infty\), we know that we
will have an encirclement if \(-\frac{K}{8}<-1\). On the positive side we also know we will have an
encirclement if \(K<-1\). Thus we want the following inequality to hold in order to have a stable system.
\[-1<K<8\]

\section*{Problem 2}

\paragraph{a)}

\paragraph{b)}

\paragraph{c)}

\paragraph{d)}

\section*{Problem 3}

\paragraph{a)}

\paragraph{b)}

\paragraph{c)}

\paragraph{d)}

\paragraph{e)}

\section*{Problem 4}

\paragraph{a)}

\paragraph{b)}

\paragraph{c)}

\paragraph{d)}

\section*{Problem 5}

\paragraph{a)}

\paragraph{b)}

\paragraph{c)}

\paragraph{d)}

\paragraph{e)}

\paragraph{f)}

\paragraph{g)}

\end{document}