\documentclass[12pt]{article}
\usepackage{amsmath}
\begin{document}
\title{Electrical Engineering 141, Homework 2}
\date{January 25th, 2019}
\author{Michael Wu\\UID: 404751542}
\maketitle

\section*{Problem 1}

\paragraph{a)}

First we have the following.
\begin{align*}
    Y_1 &= G_2G_1(R_1 - H_1Y_1)\\
    &=\frac{G_1G_2}{1+G_1G_2H_1}R_1
\end{align*}
Then we can write the transfer function for \(Y_2\) as follows.
\begin{align*}
    Y_2 &= G_4(G_3(G_6Y_1-H_2Y_2)+ G_5G_1(R_1-H_1Y_1))\\
    (1+G_4G_3H_2)Y_2&=G_4G_3G_6Y_1+ G_4G_5G_1R_1-G_4G_5G_1H_1Y_1\\
    (1+G_4G_3H_2)Y_2&=(G_4G_3G_6-G_4G_5G_1H_1)Y_1+ G_4G_5G_1R_1\\
    \frac{Y_2}{R_1}&=\left(\frac{G_4G_3G_6-G_4G_5G_1H_1}{1+G_4G_3H_2}\right)\frac{Y_1}{R_1} + \frac{G_4G_5G_1}{1+G_4G_3H_2}
\end{align*}
Then we can plug in our earlier expression to obtain the following.
\begin{align*}
    \frac{Y_2}{R_1}&=\left(\frac{G_4G_3G_6-G_4G_5G_1H_1}{1+G_4G_3H_2}\right)\frac{G_1G_2}{1+G_1G_2H_1}+\frac{G_4G_5G_1}{1+G_4G_3H_2}\\
    &=\frac{G_1G_2G_4G_3G_6-G_2G_4G_5G_1^2H_1 + G_4G_5G_1 + G_2G_4G_5G_1^2H_1}{(1+G_4G_3H_2)(1+G_1G_2H_1)}\\
    &=\frac{G_1G_2G_3G_4G_6 + G_1G_4G_5}{(1+G_3G_4H_2)(1+G_1G_2H_1)}
\end{align*}

\paragraph{b)}

If we let
\[G_5(s)=-G_2G_3G_6\]
then the previous transfer function will be zero.

\section*{Problem 2}

\paragraph{a)}

Across the first resistor we have
\[V_i-V_a = I_aZ_1\]
due to Ohm's law. Across the second resistor we also have
\[V_o-V_a = I_bZ_2\]
due to Ohm's law. Since the impedance of the OpAmp is infinity, we know that the current into the negative side is zero.
Then Kirchhoff's law gives us
\[I_a + I_b = 0\]
which we can rewrite using the previous equations to obtain the following.
\begin{align*}
    \frac{V_i-V_a}{Z_1} + \frac{V_o-V_a}{Z_2} &= 0\\
    (V_i-V_a)*Z_2 + (V_o-V_a)*Z_1 &= 0\\
    Z_2V_i + Z_1V_o &= (Z_1+Z_2)V_a\\
    V_a &= \frac{Z_2}{Z_1+Z_2}V_i + \frac{Z_1}{Z_1+Z_2}V_o
\end{align*}

\paragraph{b)}

\begin{align*}
    V_o&=-a\left(\frac{Z_2}{Z_1+Z_2}V_i + \frac{Z_1}{Z_1+Z_2}V_o\right)\\
    \left(1+\frac{aZ_1}{Z_1+Z_2}\right)V_o&=\frac{-aZ_1}{Z_1+Z_2}V_i\\
    \frac{V_o}{V_i}&=\frac{\frac{-aZ_1}{Z_1+Z_2}}{1+\frac{aZ_1}{Z_1+Z_2}}
\end{align*}

\paragraph{c)}

This is equivalent to assuming that \(V_a=0\) in our previous calculations. Then we have across the first resistor
\[V_i = I_aZ_1\]
due to Ohm's law. Across the second resistor we also have
\[V_o = I_bZ_2\]
due to Ohm's law. Then Kirchhoff's law gives us
\[I_a + I_b = 0\]
which we can rewrite as follows.
\begin{align*}
    \frac{V_i}{Z_1} + \frac{V_o}{Z_2} &= 0\\
    \frac{V_o}{Z_2} &= -\frac{V_i}{Z_1}\\
    \frac{V_o}{V_i} &= -\frac{Z_2}{Z_1}
\end{align*}

\paragraph{d)}

Let \(I_1\) be a current flowing down across \(Z_1\). Let \(I_2\) be a current flowing to the left across \(Z_2\). Then we have
that
\[V_o-V_a = I_2Z_2\]
and that
\[V_a = I_1Z_1\]
and from Kirchoff's law we have that
\[I_2 = I_1\]
This means that we know that
\begin{align*}
    \frac{V_o-V_a}{Z_2} &= \frac{V_a}{Z_1}\\
    \frac{V_o}{Z_2} &= \frac{V_a}{Z_1} + \frac{V_a}{Z_2}\\
    V_a &= \frac{Z_1}{Z_1+Z_2}V_o
\end{align*}
Additionally across the OpAmp we know that
\begin{align*}
    V_o&=a(V_i-V_a)\\
    V_o&=a\left(V_i-\frac{Z_1}{Z_1 + Z_2}V_o\right)\\
    \left(1+\frac{aZ_1}{Z_1+Z_2}\right)V_o&=aV_i\\
    \frac{V_o}{V_i}&=\frac{a}{1+\frac{aZ_1}{Z_1+Z_2}}\\
    \frac{V_o}{V_i}&=\frac{a(Z_1+Z_2)}{(1+a)Z_1+Z_2}
\end{align*}
As the gain \(a\) becomes very large the limit of this becomes
\[\frac{V_o}{V_i}=\frac{Z_1+Z_2}{Z_1}\]

\paragraph{e)}

This is the same as the inverting amplifier that we studied earlier, except with impedances
\[Z_1=\frac{\frac{R_1}{j\omega C_1}}{R_1+\frac{1}{j \omega C_1}}\]
and
\[Z_2 = R_2+\frac{1}{j \omega C_2}\]
which can be rewritten in terms of \(s\) to be
\begin{align*}
    Z_1 &=\frac{\frac{R_1}{sC_1}}{R_1+\frac{1}{sC_1}}=\frac{R_1}{1+sR_1C_1}\\
    Z_2 &=R_2 + \frac{1}{sC_2}=\frac{1+sR_2C_2}{sC_2}
\end{align*}
Then our transfer function is given by the following
\begin{align*}
    \frac{V_0(s)}{V_i(s)} &= - \frac{(1+sR_2C_2)(1+sR_1C_1)}{sR_1C_2}\\
    &=- \frac{1+s(R_1C_1+R_2C_2)+s^2R_1C_1R_2C_2}{sR_1C_2}\\
    &=-\left(\frac{C_1}{C_2}+\frac{R_1}{R_2}\right)-R_2C_1s-\frac{1}{sR_1C_2}
\end{align*}
Therefore this is equivalent to
\begin{align*}
    \frac{V_0(s)}{V_i(s)} &=K_P + K_Ds + \frac{K_I}{s}\\
    K_P &= -\frac{C_1}{C_2}-\frac{R_1}{R_2}\\
    K_D &= -R_2C_1\\
    K_I &= -\frac{1}{R_1C_2}
\end{align*}

\paragraph{f)}

We have that
\[v_o = -\frac{R_b}{R_a}v_2\]
since there is an inverting amplifier in the middle
of the circuit. Furthermore we have that \(v_2 = \frac{Q_1}{C_1}\), and so
\[v_2^\prime = \frac{I_1}{C_1}\]
where \(I_1\) is the current going through the capacitor \(C_1\). The current going through the
resistor \(R_1\) is equal to \(I_1 + I_a\), where \(I_a\) is the current going through \(R_a\). We can solve
\begin{align*}
    v_2-v_o &= I_a(R_a+R_b)\\
    \left(1+\frac{R_b}{R_a}\right)v_2&=I_a(R_a+R_b)\\
    I_a=\frac{v_2}{R_a}
\end{align*}
to obtain the expression
\[v_1-v_2 = \left(C_1v_2^\prime + \frac{v_2}{R_a}\right)R_1\]
from the left side of the circuit. Similarly the right side of
the circuit gives us
\[-\frac{R_b}{R_a}v_2-v_3 = C_2v_3^\prime R_2\]
Solving for \(v_2^\prime\) and \(v_3^\prime\) yields the system of equations shown below.
\begin{align*}
    v_2^\prime &=\left(-\frac{1}{R_1C_1}-\frac{1}{R_aC_1}\right)v_2 + \frac{1}{R_1C_1}v_1\\
    v_3^\prime &=-\frac{R_b}{R_aR_2C_2}v_2-\frac{1}{R_2C_2}v_3
\end{align*}
which is equivalent to the State-Space form
\[\frac{dx}{dt}=\begin{bmatrix}
    -\frac{1}{R_1C_1}-\frac{1}{R_aC_1} & 0\\
    -\frac{R_b}{R_aR_2C_2} & -\frac{1}{R_2C_2}
\end{bmatrix}x + \begin{bmatrix}
    \frac{1}{R_1C_1}\\
    0
\end{bmatrix}v_1\]

\section*{Problem 3}

\paragraph{a)}

Assume that movement can only occur along the horizontal axis so we can ignore gravity. Then the position of the payload is given by
\[x(t)+L\sin(\phi(t))\]
The force \(F\) on the payload must have an equal and opposite force \(-F\) on the cart. Because this force acts along the horizontal plane, we
multiply it by \(\sin(\phi(t))\). Then adding together the forces on the cart yields
the following equation.
\[Mx^{\prime\prime}(t)=u(t)-bx^{\prime}(t)-F\sin(\phi(t))\]
The payload's movement is defined by the following equation on the horizontal axis.
\[m(x(t)+L\sin(\phi(t)))^{\prime\prime} = F\sin(\phi(t))\]
The payload's movement is defined by the following equation on the vertical axis.
\[mL\cos(\phi(t))^{\prime\prime} = F\cos(\phi(t)) + mg\]
We can eliminate \(F\) in the horizontal equation as follows.
\begin{align*}
    F\sin(\phi(t))&=-Mx^{\prime\prime}(t)+u(t)-bx^{\prime}(t)\\
    F\sin(\phi(t))&=mx^{\prime\prime}(t)+mL\cos(\phi(t))\phi^{\prime\prime}(t)-mL\sin(\phi(t))(\phi^\prime(t))^2\\
    0&=(m+M)x^{\prime\prime}(t)-u(t)+bx^{\prime}(t)\\
    &\qquad +mL\cos(\phi(t))\phi^{\prime\prime}(t)-mL\sin(\phi(t))(\phi^\prime(t))^2
\end{align*}
Then expanding the vertical equation gives the following.
\begin{align*}
    F\cos(\phi(t))&=-mL\sin(\phi(t))\phi^{\prime\prime}(t)-mL\cos(\phi(t))(\phi^\prime(t))^2-mg\\
    F\sin(\phi(t))&=-mL\sin(\phi(t))\tan(\phi(t))\phi^{\prime\prime}(t)-mL\sin(\phi(t))(\phi^\prime(t))^2\\
    &\qquad- mg\tan(\phi(t))\\
    0&=-mx^{\prime\prime}(t)-mL\cos(\phi(t))\phi^{\prime\prime}(t)+mL\sin(\phi(t))(\phi^\prime(t))^2\\
    &\qquad -mL\sin(\phi(t))\tan(\phi(t))\phi^{\prime\prime}(t)-mL\sin(\phi(t))(\phi^\prime(t))^2\\
    &\qquad - mg\tan(\phi(t))\\
    0&=-mx^{\prime\prime}(t)-mL\cos(\phi(t))\phi^{\prime\prime}(t)-mL\sin(\phi(t))\tan(\phi(t))\phi^{\prime\prime}(t)\\
    &\qquad- mg\tan(\phi(t))\\
    0&=-mx^{\prime\prime}(t)\cos(\phi(t)) - mL\phi^{\prime\prime}(t)(\cos^2(\phi(t))+\sin^2(\phi(t)))\\
    &\qquad- mg\sin(\phi(t))\\
    0&=-mx^{\prime\prime}(t)\cos(\phi(t)) - mL\phi^{\prime\prime}(t)-mg\cos(\phi(t))\\
    0&=x^{\prime\prime}(t)\cos(\phi(t))+L\phi^{\prime\prime}(t)+g\sin(\phi(t))
\end{align*}

\paragraph{b)}

With \(\phi\) being small, we can approximate \(\sin(\phi)\) with \(\phi\) and \(\cos(\phi)\) with 1. Then our system of equations
becomes
\begin{align*}
    (m+M)x^{\prime\prime}(t)+mL\phi^{\prime\prime}(t)-mL\phi(t)(\phi^\prime(t))^2&=u(t)\\
    x^{\prime\prime}(t)+L\phi^{\prime\prime}(t)+g\phi(t)&=0
\end{align*}
This still contains a \((\phi^\prime(t))^2\) term which is not linear, so we assume that \(\phi\) changes slowly in order to make this
term negligible. Then our linear system is
\begin{align*}
    (m+M)x^{\prime\prime}(t)+mL\phi^{\prime\prime}(t)&=u(t)\\
    x^{\prime\prime}(t)+L\phi^{\prime\prime}(t)+g\phi(t)&=0
\end{align*}
Using the second equation and rewriting it as
\[v^\prime(t)+L\phi^{\prime\prime}(t)+g\phi(t)=0\]
we can obtain the transfer function
\begin{align*}
    sV(s)+s^2L\Phi(s)+g\Phi(s)&=0\\
    sV(s)&=-(s^2L+g)\Phi(s)\\
    \frac{\Phi(s)}{V(s)}&=-\frac{s}{s^2L+g}
\end{align*}

\paragraph{c)}

If \(v(t)\) is the unit step function, then it has a Laplace transform of \(\frac{1}{s}\). Then we have
\[\Phi(s)=-\frac{1}{s^2L+g}=-\frac{1}{\sqrt{gL}}\frac{\sqrt{\frac{g}{L}}}{s^2+\frac{g}{L}}\]
which has an inverse Laplace transform
\[\phi(t)=-\frac{\sin\left(\sqrt{\frac{g}{L}}t\right)}{\sqrt{gL}}\]
so it does oscillate with frequency \(\omega_0=\sqrt{\frac{g}{L}}\).

\paragraph{d)}

Taking the Laplace transform of our system of equations yields the following
\begin{align*}
    (m+M)s^2X(s)+mLs^2\Phi(s)&=U(s)\\
    s^2X(s)+Ls^2\Phi(s)+g\Phi(s)&=0\\
    \Phi(s)&=-\frac{s^2}{Ls^2+g}X(s)\\
    (m+M)s^2X(s)-\frac{mLs^4}{Ls^2+g}X(s)&=U(s)\\
    \frac{(m+M)s^2(Ls^2+g)-mLs^4}{Ls^2+g}X(s)&=U(s)\\
    \frac{s^2(MLs^2+(m+M)g)}{Ls^2+g}X(s)&=U(s)\\
    \frac{X(s)}{U(s)}&=\frac{Ls^2+g}{s^2(MLs^2+(m+M)g)}
\end{align*}

\paragraph{e)}

If \(u(t)\) is the unit step function, then it has a Laplace transform of \(\frac{1}{s}\). Then we have
\[X(s)=\frac{Ls^2+g}{s^3(MLs^2+(m+M)g)}\]
We can try using the final value theorem by taking the following limit.
\[\lim_{s\to 0} sX(s)=\lim_{s\to 0} \frac{Ls^2+g}{s^2(MLs^2+(m+M)g)}\]
The denominator goes to \(0\) while the numerator goes to \(g\), so this limit does not exist and the value goes to \(\infty\).
Therefore by the final value theorem no limit exists and the system's position increases without bound.

\section*{Problem 4}

We can convert each equation into a linear approximation by taking the appropriate derivatives evaluated at our initial point.
For the first equation we have
\[x_1^\prime \approx u\frac{\partial f_1}{\partial u} + x_2\frac{\partial f_1}{\partial x_2}=0 + x_2\]
For the second equation we have
\[x_2^\prime \approx u\frac{\partial f_2}{\partial u} + x_4\frac{\partial f_2}{\partial x_4}=-\pi u - \pi x_4\]
For the third equation we have
\[x_3^\prime \approx u\frac{\partial f_3}{\partial u} + x_2\frac{\partial f_3}{\partial x_2}=\pi u + \pi x_2\]
For the fourth equation we have
\[x_4^\prime \approx x_1\frac{\partial f_4}{\partial x_1} + x_3\frac{\partial f_4}{\partial x_3}=2x_1 + 4x_3\]
This yields the linear system
\[\delta x^\prime = \begin{bmatrix}
    0 & 1 & 0 & 0\\
    0 & 0 & 0 & -\pi\\
    0 & \pi & 0 & 0\\
    2 & 0 & 4 & 0
\end{bmatrix}\delta x + \begin{bmatrix}
    0\\
    -\pi\\
    \pi\\
    0
\end{bmatrix}\delta u\]

\section*{Problem 5}

The controller canonical form is shown below. The state is given by the following.
\[\begin{bmatrix}
    x_1^\prime\\
    x_2^\prime\\
    x_3^\prime
\end{bmatrix}=\begin{bmatrix}
    0 & 1 & 0\\
    0 & 0 & 1\\
    -\frac{K_0}{K_3} & -\frac{K_1}{K_3} & -\frac{K_2}{K_3}
\end{bmatrix}\begin{bmatrix}
    x_1\\
    x_2\\
    x_3
\end{bmatrix} + \begin{bmatrix}
    0\\
    0\\
    \frac{1}{K_3}
\end{bmatrix}\delta(t)
\]
The output is given by the following.
\[\phi(t) = \begin{bmatrix} K_b & K_a & 0\end{bmatrix}\begin{bmatrix}
    x_1\\
    x_2\\
    x_3
\end{bmatrix}\]

\section*{Problem 6}

\paragraph{a)}

\begin{align*}
    sX(s)&=AX(s)+BU(s)\\
    (sI-A)X(s)&=BU(s)\\
    X(s)&=(sI-A)^{-1}BU(s)\\
    Y(s)&=C(sI-A)^{-1}BU(s)\\
    \frac{Y(s)}{U(s)}&=C(sI-A)^{-1}B
\end{align*}

\paragraph{b)}

The transfer function in the form
\[H(s)=K\frac{\Pi_i (s-z_i)}{\Pi_j(s-p_j)}\]
has the constants
\begin{align*}
    K&=0.0003\\
    z_1&=-0.394\\
    z_2&=-0.02\\
    p_1&=-0.656\\
    p_2&=-0.1889\\
    p_3&=-0.0042
\end{align*}

\section*{Problem 7}

\paragraph{a)}

\begin{align*}
    (s-1)(s-2)-1 &= 0\\
    s^2-3s+1 &= 0\\
    s&=\frac{3\pm\sqrt{5}}{2}
\end{align*}
One of these poles is positive and one is negative, so it is not stable. As time increases the positive pole will
cause exponential growth.

\paragraph{b)}

\begin{align*}
    (s-1+k_1)(s-2)+(k_2-1) &= (s+2)^2 + 1\\
    s^2-3s+k_1s+k_2-1 &= s^2+4s+5
\end{align*}
This gives us \(k_1=7\) and \(k_2=6\).

\paragraph{c)}

No you cannot stabilize the system in this case. The characteristic equation will always contain a term \(s-2\) which means
that a positive pole exists, leading to instability. The gain term \(k_2\) will have no effect on the pole locations since the
0 in the bottom left makes it have no impact on the characteristic equation.

\end{document}