\documentclass[12pt]{article}
\usepackage{amsmath}
\begin{document}
\title{Electrical Engineering 141, Homework 1}
\date{January 25th, 2019}
\author{Michael Wu\\UID: 404751542}
\maketitle

\section*{Problem 1}

\paragraph{a)}

First we have the following.
\begin{align*}
    Y_1 &= G_2G_1(R_1 - H_1Y_1)\\
    &=\frac{G_1G_2}{1+G_1G_2H_1}R_1
\end{align*}
Then we can write the transfer function for \(Y_2\) as follows.
\begin{align*}
    Y_2 &= G_4(G_3(G_6Y_1-H_2Y_2)+ G_5G_1(R_1-H_1Y_1))\\
    (1+G_4G_3H_2)Y_2&=G_4G_3G_6Y_1+ G_4G_5G_1R_1-G_4G_5G_1H_1Y_1\\
    (1+G_4G_3H_2)Y_2&=(G_4G_3G_6-G_4G_5G_1H_1)Y_1+ G_4G_5G_1R_1\\
    \frac{Y_2}{R_1}&=\left(\frac{G_4G_3G_6-G_4G_5G_1H_1}{1+G_4G_3H_2}\right)\frac{Y_1}{R_1} + \frac{G_4G_5G_1}{1+G_4G_3H_2}
\end{align*}
Then we can plug in our earlier expression to obtain the following.
\begin{align*}
    \frac{Y_2}{R_1}&=\left(\frac{G_4G_3G_6-G_4G_5G_1H_1}{1+G_4G_3H_2}\right)\frac{G_1G_2}{1+G_1G_2H_1}+\frac{G_4G_5G_1}{1+G_4G_3H_2}\\
    &=\frac{G_1G_2G_4G_3G_6-G_2G_4G_5G_1^2H_1 + G_4G_5G_1 + G_2G_4G_5G_1^2H_1}{(1+G_4G_3H_2)(1+G_1G_2H_1)}\\
    &=\frac{G_1G_2G_3G_4G_6 + G_1G_4G_5}{(1+G_3G_4H_2)(1+G_1G_2H_1)}
\end{align*}

\paragraph{b)}

If we let
\[G_5(s)=-G_2G_3G_6\]
then the previous transfer function will be zero.

\section*{Problem 2}

\paragraph{a)}

Across the first resistor we have
\[V_i-V_a = I_aZ_1\]
due to Ohm's law. Across the second resistor we also have
\[V_o-V_a = I_bZ_2\]
due to Ohm's law. Since the impedance of the OpAmp is infinity, we know that the current into the negative side is zero.
Then Kirchhoff's law gives us
\[I_a + I_b = 0\]
which we can rewrite using the previous equations to obtain the following.
\begin{align*}
    \frac{V_i-V_a}{Z_1} + \frac{V_o-V_a}{Z_2} &= 0\\
    (V_i-V_a)*Z_2 + (V_o-V_a)*Z_1 &= 0\\
    Z_2V_i + Z_1V_o &= (Z_1+Z_2)V_a\\
    V_a &= \frac{Z_2}{Z_1+Z_2}V_i + \frac{Z_1}{Z_1+Z_2}V_o
\end{align*}

\paragraph{b)}

\begin{align*}
    V_o&=-a\left(\frac{Z_2}{Z_1+Z_2}V_i + \frac{Z_1}{Z_1+Z_2}V_o\right)\\
    \left(1+\frac{aZ_1}{Z_1+Z_2}\right)V_o&=\frac{-aZ_1}{Z_1+Z_2}V_i\\
    \frac{V_o}{V_i}&=\frac{\frac{-aZ_1}{Z_1+Z_2}}{1+\frac{aZ_1}{Z_1+Z_2}}
\end{align*}

\paragraph{c)}

This is equivalent to assuming that \(V_a=0\) in our previous calculations. Then we have across the first resistor
\[V_i = I_aZ_1\]
due to Ohm's law. Across the second resistor we also have
\[V_o = I_bZ_2\]
due to Ohm's law. Then Kirchhoff's law gives us
\[I_a + I_b = 0\]
which we can rewrite as follows.
\begin{align*}
    \frac{V_i}{Z_1} + \frac{V_o}{Z_2} &= 0\\
    \frac{V_o}{Z_2} &= -\frac{V_i}{Z_1}\\
    \frac{V_o}{V_i} &= -\frac{Z_2}{Z_1}
\end{align*}

\paragraph{d)}

Let \(I_1\) be a current flowing down across \(Z_1\). Let \(I_2\) be a current flowing to the left across \(Z_2\). Then we have
that
\[V_o-V_a = I_2Z_2\]
and that
\[V_a = I_1Z_1\]
and from Kirchoff's law we have that
\[I_2 = I_1\]
This means that we know that
\begin{align*}
    \frac{V_o-V_a}{Z_2} &= \frac{V_a}{Z_1}\\
    \frac{V_o}{Z_2} &= \frac{V_a}{Z_1} + \frac{V_a}{Z_2}\\
    V_a &= \frac{Z_1}{Z_1+Z_2}V_o
\end{align*}
Additionally across the OpAmp we know that
\begin{align*}
    V_o&=a(V_i-V_a)\\
    V_o&=a\left(V_i-\frac{Z_1}{Z_1 + Z_2}V_o\right)\\
    \left(1+\frac{aZ_1}{Z_1+Z_2}\right)V_o&=aV_i\\
    \frac{V_o}{V_i}&=\frac{a}{1+\frac{aZ_1}{Z_1+Z_2}}\\
    \frac{V_o}{V_i}&=\frac{a(Z_1+Z_2)}{(1+a)Z_1+Z_2}
\end{align*}
As the gain \(a\) becomes very large the limit of this becomes
\[\frac{V_o}{V_i}=\frac{Z_1+Z_2}{Z_1}\]

\paragraph{e)}

This is the same as the inverting amplifier that we studied earlier, except with impedances
\[Z_1=\frac{\frac{R_1}{j\omega C_1}}{R_1+\frac{1}{j \omega C_1}}\]
and
\[Z_2 = R_2+\frac{1}{j \omega C_2}\]
which can be rewritten in terms of \(s\) to be
\begin{align*}
    Z_1 &=\frac{\frac{R_1}{sC_1}}{R_1+\frac{1}{sC_1}}=\frac{R_1}{1+sR_1C_1}\\
    Z_2 &=R_2 + \frac{1}{sC_2}=\frac{1+sR_2C_2}{sC_2}
\end{align*}
Then our transfer function is given by the following
\begin{align*}
    \frac{V_0(s)}{V_i(s)} &= - \frac{(1+sR_2C_2)(1+sR_1C_1)}{sR_1C_2}\\
    &=- \frac{1+s(R_1C_1+R_2C_2)+s^2R_1C_1R_2C_2}{sR_1C_2}\\
    &=-\left(\frac{C_1}{C_2}+\frac{R_1}{R_2}\right)-R_2C_1s-\frac{1}{sR_1C_2}
\end{align*}
Therefore this is equivalent to
\begin{align*}
    \frac{V_0(s)}{V_i(s)} &=K_P + K_Ds + \frac{K_I}{s}\\
    K_P &= -\frac{C_1}{C_2}-\frac{R_1}{R_2}\\
    K_D &= -R_2C_1\\
    K_I &= -\frac{1}{R_1C_2}
\end{align*}

\paragraph{f)}

We have that
\[v_o = -\frac{R_b}{R_a}v_2\]
since there is an inverting amplifier in the middle
of the circuit. Furthermore we have that\(v_2 = \frac{Q_1}{C_1}\), and so
\[v_2^\prime = \frac{I_1}{C_1}\]
where \(I_1\) is the current going through the capacitor \(C_1\). The current going through the
resistor \(R_1\) is equal to \(I_1 + I_a\), where \(I_a\) is the current going through \(R_a\). We can solve
\begin{align*}
    v_2-v_o &= I_a(R_a+R_b)\\
    \left(1+\frac{R_b}{R_a}\right)v_2&=I_a(R_a+R_b)\\
    I_a=\frac{v_2}{R_a}
\end{align*}
to obtain the expression
\[v_1-v_2 = \left(C_1v_2^\prime + \frac{v_2}{R_a}\right)R_1\]
from the left side of the circuit. Similarly the right side of
the circuit gives us
\[-\frac{R_b}{R_a}v_2-v_3 = C_2v_3^\prime R_2\]
Solving for \(v_2^\prime\) and \(v_3^\prime\) yields the system of equations shown below.
\begin{align*}
    v_2^\prime &=\left(-\frac{1}{R_1C_1}-\frac{1}{R_aC_1}\right)v_2 + \frac{1}{R_1C_1}v_1\\
    v_3^\prime &=-\frac{R_b}{R_aR_2C_2}v_2-\frac{1}{R_2C_2}v_3
\end{align*}
which is equivalent to the State-Space form
\[\frac{dx}{dt}=\begin{bmatrix}
    -\frac{1}{R_1C_1}-\frac{1}{R_aC_1} & 0\\
    -\frac{R_b}{R_aR_2C_2} & -\frac{1}{R_2C_2}
\end{bmatrix}x + \begin{bmatrix}
    \frac{1}{R_1C_1}\\
    0
\end{bmatrix}v_1\]

\section*{Problem 3}

\paragraph{a)}

\paragraph{b)}

\paragraph{c)}

\paragraph{d)}

\paragraph{e)}

\section*{Problem 4}

We can convert each equation into a linear approximation by taking the appropriate derivatives evaluated at our initial point.
For the first equation we have
\[x_1^\prime \approx u\frac{\partial f_1}{\partial u} + x_2\frac{\partial f_1}{\partial x_2}=0 + x_2\]
For the second equation we have
\[x_2^\prime \approx u\frac{\partial f_2}{\partial u} + x_4\frac{\partial f_2}{\partial x_4}=-\pi u - \pi x_4\]
For the third equation we have
\[x_3^\prime \approx u\frac{\partial f_3}{\partial u} + x_2\frac{\partial f_3}{\partial x_2}=\pi u + \pi x_2\]
For the fourth equation we have
\[x_4^\prime \approx x_1\frac{\partial f_4}{\partial x_1} + x_3\frac{\partial f_4}{\partial x_3}=2x_1 + 4x_3\]
This yields the linear system
\[\delta x^\prime = \begin{bmatrix}
    0 & 1 & 0 & 0\\
    0 & 0 & 0 & -\pi\\
    0 & \pi & 0 & 0\\
    2 & 0 & 4 & 0
\end{bmatrix}\delta x + \begin{bmatrix}
    0\\
    -\pi\\
    \pi\\
    0
\end{bmatrix}\delta u\]

\section*{Problem 5}

\section*{Problem 6}

\paragraph{a)}

\paragraph{b)}

\section*{Problem 7}

\paragraph{a)}

\paragraph{b)}

\paragraph{c)}

\end{document}