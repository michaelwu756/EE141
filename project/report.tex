\documentclass[12pt]{article}
\usepackage{amsmath}
\usepackage{mathtools}
\begin{document}
\title{Electrical Engineering 141, Final Project Report}
\date{March 14th, 2019}
\author{Michael Wu\\UID: 404751542}
\maketitle

\section{Current Control Loop Design}

\paragraph{a)}

Consider the relationship between \(V_a\) and \(V_c\). The voltage at the negative
terminal of the OpAmp is \(\frac{V_a}{5}\). Then we can solve the following equation.
\begin{align*}
    V_a&=k\left(V_c-\frac{V_a}{5}\right)\\
    V_a&=\frac{5k}{5+k}V_c\\
    V_a&=5V_c
\end{align*}

Next consider the relationship between \(V_s\) and \(I_c\). Since \(R_s\) is very low,
we can assume all the current flows through \(R_s\). Then the voltage at the positive input of the
OpAmp is \(\frac{5}{6}I_cR_s\). Since \(R_s=0.2\) this just becomes \(\frac{I_c}{6}\).
The voltage at the negative input of the OpAmp is \(\frac{V_s}{6}\).
We then have the following relationship.
\begin{align*}
    V_s&=\frac{k}{6}(I_c-V_s)\\
    V_s&=\frac{k}{6+k}I_c\\
    V_s&=I_c
\end{align*}

Now consider the term \(x^\prime\). This has a Laplace transform of \(sX\).
From the mechanical behavior of the FTS, we obtain the following equation.
\begin{align*}
    s^2m_1X &= K_fI_c - sb_1X - k_1X\\
    X&=\frac{K_f}{m_1s^2 + b_1s + k_1}I_c\\
    sX&=\frac{K_fs}{m_1s^2 + b_1s + k_1}I_c
\end{align*}
Then using \(I_c\) and \(V_a\) we can then solve the following relationship.
\begin{align*}
    I_c (R_c + R_s + sL_c) &= V_a - K_f sX\\
    I_c (R_c + R_s + sL_c) &= V_a - \frac{K_f^2s}{m_1s^2 + b_1s + k_1}I_c\\
    I_c &=\frac{m_1s^2 + b_1s + k_1}{(m_1s^2 + b_1s + k_1)(R_c + R_s + sL_c)+K_f^2s}V_a\\
    V_s &= \frac{5(m_1s^2 + b_1s + k_1)}{(m_1s^2 + b_1s + k_1)(R_c + R_s + sL_c)+K_f^2s}V_c\\
    P_{elec} &= \frac{5(m_1s^2 + b_1s + k_1)}{(m_1s^2 + b_1s + k_1)(4.2 + 0.0002s)+K_f^2s}
\end{align*}

\paragraph{b)}

If the back emf is zero, then we can obtain the following equation instead.
\begin{align*}
    I_c (R_c + R_s + sL_c) &= V_a\\
    I_c &=\frac{1}{R_c + R_s + sL_c}V_a\\
    V_s &= \frac{5}{R_c + R_s + sL_c}V_c\\
    P_{elec} &= \frac{5}{4.2 + 0.0002s}
\end{align*}

\paragraph{c)}

At high frequencies, these are equal because the \(s^2\) term will dominate the transfer
function with back emf included. Then only the coefficient \(5m_1\) in the numerator and
\((4.2 + 0.0002s)m_1\) in the denominator will have an effect on the transfer function.
Thus we obtain the following transfer function.
\[P_{elec} \approx \frac{5m_1}{(4.2 + 0.0002s)m_1} = \frac{5}{4.2 + 0.0002s}\]
This is exactly the same as the transfer function with no back emf. The physical reason
this happens is because as the frequency increases beyond the resonant frequency, the FTS
dynamics will cause the position and velocity of the FTS to become smaller and smaller.

\paragraph{d)}

The components \(R_3\), \(C_1\), and \(C_2\) form an equivalent component with the following
impedance.
\[Z=\frac{C_1R_3s + 1}{s(C_1C_2R_3s + C_1 + C_2)}\]
We can assume the input to the negative terminal of the OpAmp is zero. Then we can write
the following equations.
\begin{align*}
    \frac{V_c}{Z}&=-\frac{V_s}{R_2}\\
    -\frac{V_c}{V_s}&=\frac{Z}{R_2}\\
    C_{elec} &=\frac{C_1R_3s + 1}{R_2s(C_1C_2R_3s + C_1 + C_2)}
\end{align*}
The transfer function between \(V_c\) and \(V_{set}\) is exactly the same
except the sign is flipped and \(R_2\) is replaced by \(R_1\). Since
\(V_c=C_{elec}H_{V_{set}V_r}V_{set}\), we then know that \(H_{V_{set}V_r}=-\frac{R_2}{R_1}\).
The overall closed loop transfer function is shown below.
\[H_{V_{set}I_c}=H_{V_{set}V_r}\frac{C_{elec}P_{elec}}{1+C_{elec}P_{elec}}\]
Note that as \(s\) becomes very small, the open loop transfer function will go to infinity due to the
root at zero. In the closed loop transfer function, this means that both the numerator and denominator
of the fractional term will grow at the same rate. This leads to a limit of one as \(s\) goes to zero.
Then with the steady state gain condition we can use the final value theorem to write the following equation.
\begin{align*}
    5&=\lim_{s\to 0} s\frac{-10}{s}H_{V_{set}V_r}\frac{C_{elec}P_{elec}}{1+C_{elec}P_{elec}}\\
    &=10\frac{R_2}{R_1}\\
    R_2&=5000
\end{align*}
If we let \(s=(6\times 10^5)j\), we can then use the gain crossover and phase margin conditions to solve for
the remaining component values. The value of the plant is the following.
\[P_{elec}=\frac{5}{4.2 + 0.0002\times6\times 10^5j}=0.04164116925e^{j\frac{\pi}{180}(-87.9955)}\]
The value of the controller is the following.
\begin{align*}
    C_{elec} &=\frac{C_1R_3(6\times 10^5)j + 1}{5000(6\times 10^5)j(C_1C_2R_3(6\times 10^5)j + C_1 + C_2)}\\
    &=\frac{1}{3\times10^9}\frac{1+jC_1R_3(6\times 10^5)}{-C_1C_2R_3(6\times 10^5)+j(C_1+C_2)}
\end{align*}
We want the gain to be 1 at this frequency so we obtain the following equation.
\begin{align*}
    \frac{0.04164116925}{3\times10^9}\sqrt{\frac{C_1^2R_3^2(6\times 10^5)^2 + 1}{C_1^2C_2^2R_3^2(6\times 10^5)^2 + (C_1+C_2)^2}} &=1\\
    \sqrt{\frac{C_1^2R_3^2(6\times 10^5)^2 + 1}{C_1^2C_2^2R_3^2(6\times 10^5)^2 + (C_1+C_2)^2}}&=\frac{3\times10^9}{0.04164116925}
\end{align*}
From the phase margin we obtain the following equation.
\begin{align*}
    \theta &= \tan^{-1}(C_1R_3(6\times 10^5))-90-\tan^{-1}\left(\frac{C_1C_2R_3(6\times 10^5)}{C_1+C_2}\right)-87.9955\\
    \theta &= \tan^{-1}(C_1R_3(6\times 10^5))-\tan^{-1}\left(\frac{C_1C_2R_3(6\times 10^5)}{C_1+C_2}\right) > 60
\end{align*}
Assume that the following conditions hold.
\begin{align*}
    C_1 &\gg C_2\\
    C_1R_3(6\times 10^5) &\gg 1\\
    C_2R_3(6\times 10^5) &\ll 1
\end{align*}
Then the first term will be near 90 and the second term will be near zero which meets our phase requirements.
Lets assume \(C_1R_3 = \frac{1}{6} \times 10^{-4}\) and \(C_1 = 100C_2\). Then we can reduce the gain
equation to obtain the following result.
\begin{align*}
    \sqrt{\frac{10^2 + 1}{C_2^210^2 + (C_1 + C_2)^2}}&=\frac{3\times10^9}{0.04164116925}\\
    \sqrt{\frac{101}{C_2^2(10^2+101^2)}}&=\frac{3\times10^9}{0.04164116925}\\
    \frac{1}{C_2}&=\sqrt{\frac{10^2+101^2}{101}}\frac{3\times10^9}{0.04164116925}\\
    C_2&=\sqrt{\frac{101}{10^2+101^2}}\frac{0.04164116925}{3\times10^9}=1.374430089\times 10^{-12}
\end{align*}
By substituting into previous conditions we then get \(C_1=1.374430089\times 10^{-10}\) and \(R_3=1.212623821\times10^5\).

\section{FTS Plant System Identification}

\section{Position Control Loop Design}

\end{document}