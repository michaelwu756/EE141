\documentclass[12pt]{article}
\usepackage{amsmath}
\begin{document}
\title{Electrical Engineering 141, Homework 7}
\date{March 18th, 2019}
\author{Michael Wu\\UID: 404751542}
\maketitle

\section*{Problem 1}

\paragraph{a)}

\begin{align*}
    \left(1-\frac{1}{2}z^{-1}\right)\left(1+\frac{1}{3}z^{-1}\right)Y(z)&=\left(1+\frac{1}{2}z^{-1}\right)U(z)\\
    \left(1+\left(\frac{1}{3}-\frac{1}{2}\right)z^{-1}-\frac{1}{6}z^{-2}\right)Y(z)&=\left(1+\frac{1}{2}z^{-1}\right)U(z)\\
    y(k)-\frac{1}{6}y(k-1)-\frac{1}{6}y(k-2)&=u(k)+\frac{1}{2}u(k-1)
\end{align*}

\paragraph{b)}

For the pole at \(z=\frac{1}{2}\) we can find the associated continuous pole with the following equation.
\begin{align*}
    z&=e^{sT}\\
    sT&=\ln\left(\frac{1}{2}\right)\\
    s&=-\frac{\ln(2)}{T}
\end{align*}
Since the poles are also given by \(s=\omega_n(-\zeta\pm\sqrt{\zeta^2-1})\), we can solve this for \(\omega_n\)
and \(\zeta\) to obtain \(\omega_n=\frac{\ln(2)}{T}\) and \(\zeta=1\). We have \(T=1\) so \(\omega_n=\ln(2)\).

For the pole at \(z=-\frac{1}{3}\) we can find the associated continuous pole with the following equation.
\begin{align*}
    z&=e^{sT}\\
    sT&=\ln\left(-\frac{1}{3}\right)\\
    s&=-\frac{\ln(-3)}{T}\\
    s&=-\frac{\ln(3)+j\pi}{T}
\end{align*}
We can then solve for \(\omega_n\) and \(\zeta\) as follows.
\begin{align*}
    \omega_n(-\zeta\pm\sqrt{\zeta^2-1})&=-\ln(3)-j\pi\\
    \omega_n&=\sqrt{\ln(3)^2+\pi^2}\\
    \zeta&=\frac{\ln(3)}{\sqrt{\ln(3)^2+\pi^2}}
\end{align*}

\paragraph{c)}

This is stable because both poles are within the unit circle.

\section*{Problem 2}

We have the following transfer function.
\begin{align*}
    (1-3z^{-1}+2z^{-2})Y(z)&=(2z^{-1}-2z^{-2})U(z)\\
    \frac{Y(z)}{U(z)}&=\frac{2z^{-1}-2z^{-2}}{1-3z^{-1}+2z^{-2}}
\end{align*}
The Z-transform of \(u(k)\) is \(\frac{z}{(z-1)^2}\). Then we have the following result.
\begin{align*}
    Y(z)&=\frac{2z^{-1}-2z^{-2}}{1-3z^{-1}+2z^{-2}}\frac{z}{(z-1)^2}\\
    &=\frac{2z-2}{z^2-3z+2}\frac{z}{(z-1)^2}\\
    &=\frac{z(2z-2)}{(z-1)^3(z-2)}\\
    &=\frac{2z}{(z-1)^2(z-2)}\\
    &=\frac{2z}{z-2} - \frac{2z}{z-1} - \frac{2z}{(z-1)^2}\\
    y(k)&=2^{k+1} - 2 - 2k
\end{align*}

\end{document}