\documentclass[12pt]{article}
\usepackage{amsmath}
\usepackage{graphicx}
\usepackage{float}
\begin{document}
\title{Electrical Engineering 141, Homework 5}
\date{February 22nd, 2019}
\author{Michael Wu\\UID: 404751542}
\maketitle

\section*{Problem 1}

\paragraph{a)}

This can be rewritten as follows.
\[L(s)=\frac{1}{200}\times\frac{1}{s}\times\frac{1}{s+1}\times\frac{1}{\frac{1}{100}s+1}\times\left(\frac{1}{5}s+1\right)\times\left(\frac{1}{10}s+1\right)\]
My asymptote sketch and the actual bode plot is shown below.
\begin{figure}[H]
    \begin{center}
        \includegraphics[width=2.5in]{problem1a.jpg}
        \includegraphics[width=2.5in]{problem1a.pdf}
    \end{center}
\end{figure}

\paragraph{b)}

This can be rewritten as follows.
\[L(s)=\frac{1}{s+1}\times\frac{1}{-s+1}\times s\]
The \(\frac{1}{-s+1}\) term produces the same magnitude as the first term but its phase goes to \(90\) instead of \(-90\) degrees since the imaginary component is reversed.
My asymptote sketch and the actual bode plot is shown below.
\begin{figure}[H]
    \begin{center}
        \includegraphics[width=2.5in]{problem1b.jpg}
        \includegraphics[width=2.5in]{problem1b.pdf}
    \end{center}
\end{figure}

\paragraph{c)}

This can be rewritten as follows.
\[L(s)=\frac{1}{s}\times\frac{1}{s+1}\times(-s+1)\]
The \(-s+1\) term has a phase that goes from 0 degrees to \(-90\) degrees since the imaginary component is reversed.
My asymptote sketch and the actual bode plot is shown below.
\begin{figure}[H]
    \begin{center}
        \includegraphics[width=2.5in]{problem1c.jpg}
        \includegraphics[width=2.5in]{problem1c.pdf}
    \end{center}
\end{figure}

\paragraph{d)}

This can be rewritten as follows.
\[L(s)=4\times\frac{1}{s}\times\frac{1}{s^2+1}\times\left(\left(\frac{s}{2}\right)^2+1\right)\]
The \(\frac{1}{s^2+1}\) term will have a phase that jumps immediately from 0 to \(-180\) after \(\omega=1\). Similarly
the quadratic term in the numerator will have a phase that jumps from 0 to \(180\) after \(\omega=2\). It has a magnitude asymptote
that increases at 40 decibels every decade after this point as well.
My asymptote sketch and the actual bode plot is shown below.
\begin{figure}[H]
    \begin{center}
        \includegraphics[width=2.5in]{problem1d.jpg}
        \includegraphics[width=2.5in]{problem1d.pdf}
    \end{center}
\end{figure}

\paragraph{e)}

This can be rewritten as follows.
\[L(s)=\frac{1}{10}\times\frac{1}{s^3}\times\frac{1}{\frac{1}{10}s+1}\times(s+1)\times(s+1)\]
My asymptote sketch and the actual bode plot is shown below.
\begin{figure}[H]
    \begin{center}
        \includegraphics[width=2.5in]{problem1e.jpg}
        \includegraphics[width=2.5in]{problem1e.pdf}
    \end{center}
\end{figure}

\paragraph{f)}

This can be rewritten as follows.
\[L(s)=e^{-0.2s}\times\frac{1}{s}\times\frac{1}{s+1}\]
The exponential term has no effect on the magnitude. It will cause the phase to rapidly move cycle through all the angles as the frequency
increases, so there is no use predicting the asymptote for frequencies greater than 1.
My asymptote sketch and the actual bode plot is shown below.
\begin{figure}[H]
    \begin{center}
        \includegraphics[width=2.5in]{problem1f.jpg}
        \includegraphics[width=2.5in]{problem1f.pdf}
    \end{center}
\end{figure}

\paragraph{g)}

This can be rewritten as follows.
\[L(s)=\frac{1}{10}\times\frac{1}{s}\times\frac{100}{s^2+20s+100}\times(s+1)\]
My asymptote sketch and the actual bode plot is shown below.
\begin{figure}[H]
    \begin{center}
        \includegraphics[width=2.5in]{problem1g.jpg}
        \includegraphics[width=2.5in]{problem1g.pdf}
    \end{center}
\end{figure}

\paragraph{h)}
This can be rewritten as follows.
\[L(s)=\frac{1}{s^2}\times\left(\frac{1}{2}s+1\right)\]
My asymptote sketch and the actual bode plot is shown below.
\begin{figure}[H]
    \begin{center}
        \includegraphics[width=2.5in]{problem1h.jpg}
        \includegraphics[width=2.5in]{problem1h.pdf}
    \end{center}
\end{figure}

\section*{Problem 2}

\paragraph{a)}

\paragraph{b)}

\paragraph{c)}

\paragraph{d)}

\paragraph{e)}

\section*{Problem 3}

\paragraph{a)}

\paragraph{b)}

\paragraph{c)}

\paragraph{d)}

\paragraph{e)}

\section*{Problem 4}

\paragraph{a)}

\paragraph{b)}

\paragraph{c)}

\paragraph{d)}

\section*{Problem 5}

\paragraph{a)}

Experimentally I used MATLAB to obtain the following transfer function.
\[H(s)=\frac{400\left(\left(\frac{s}{10}\right)+1\right)\left(\left(\frac{s}{10000}\right)+1\right)}{(s+1)\left(\left(\frac{s}{100}\right)+1\right)\left(\left(\frac{s}{1000}\right)+1\right)}\]

\paragraph{b)}

The system is type 0, since there are no pure integrators. This is apparent since a pure integrator would cause the bode plot to have a negative slope on the left side, but this
bode plot is flat. The steady state value of the open loop system is \(K_p=400\) from the final value theorem. Then the error constant for the closed loop
system is also \(400\).

\paragraph{c)}

The steady state tracking error for a step input in a unity feedback loop would be \(\frac{1}{1+K_p}=\frac{1}{401}\).

\section*{Problem 6}

\paragraph{a)}

We can calculate \(\zeta\) from the peak gain.
\begin{align*}
    \frac{1}{2\zeta\sqrt{1-\zeta^2}}&=10^{0.6}\\
    \zeta&=0.126613
\end{align*}
We have \(\omega_r=0.9=\omega_n\sqrt{1-2\zeta^2}\). Then we can solve for \(\omega_n\).
\begin{align*}
    \omega_n&=\frac{0.9}{\sqrt{1-2\zeta^2}}\\
    &=0.91478
\end{align*}
Then the best second order approximation for this system is the following.
\[H(s)=\frac{0.83682}{s^2+0.231647s+0.83682}\]

\paragraph{b)}

Using MATLAB I obtained the system bandwidth \(\omega=1.4047\).

\paragraph{c)}

\[e^{-\frac{\pi\zeta}{\sqrt{1-\zeta^2}}}=0.66965\]
The percentage overshoot is approximately 66.965\%.

\end{document}