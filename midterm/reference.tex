\documentclass[10pt]{article}
\usepackage[margin=0.5in]{geometry}
\usepackage{amsmath}
\begin{document}
\begin{enumerate}
    \item How to find equilibrium points for nonlinear systems and
    how to linearize them around a given equilibrium point.

    \(F(x,u)=0\) and linearize with the Jacobian.

    \item How to determine stability from the state-space representation
    and how to go from state-space to transfer function and vice versa.

    Given the following transfer function.
    \[\frac{b_ns^n+\ldots+b_0}{a_ns^n+\ldots+a_0}\]
    We have the following controller canonical form.
    \[\mathbf{x}^\prime=\begin{bmatrix}
        0 & 1 & \cdots & 0\\
        \vdots & \vdots & \ddots & \vdots\\
        0 & 0 & \cdots & 1\\
        -\frac{a_0}{a_n} & -\frac{a_1}{a_n} & \cdots & -\frac{a_{n-1}}{a_n}
    \end{bmatrix}\mathbf{x}
    + \begin{bmatrix} 0\\ 0\\ \vdots\\ \frac{1}{a_n}\end{bmatrix}\mathbf{u}\]
    \[\mathbf{y}=\begin{bmatrix}
        b_0-\frac{b_na_0}{a_n} & \cdots &  b_{n-1}-\frac{b_na_{n-1}}{a_n}
    \end{bmatrix}\mathbf{x} + \frac{b_n}{a_n}\mathbf{u}\]
    \begin{align*}
        sX(s)&=AX(s)+BU(s)\\
        (sI-A)X(s)&=BU(s)\\
        X(s)&=(sI-A)^{-1}BU(s)\\
        Y(s)&=C(sI-A)^{-1}BU(s)+DU(s)\\
        \frac{Y(s)}{U(s)}&=C(sI-A)^{-1}B+D
    \end{align*}
    The eigenvalues of \(A\) are the poles of the transfer function.

    \item How to go from ordinary differential equations to transfer functions and vice versa.

    Use the derivative rule of the Laplace transform.

    \item How to find various input/output relationships in feedback systems.
    \item How to find the steady-state response of LTI systems to sinusoidal inputs using the
    fundamental properties of transfer functions.

    Final value theorem.

    \item How the Internal Model Principle can help you design proper compensators in order to
    track a given reference input with zero steady-state error and/or reject the given disturbances
    at the output in the steady-state.

    The open loop transfer function \(C(s)P(s)\) must have the input's closed right hand plane poles
    and the compensator \(C(s)\) must have the disturbance's closed right hand plane poles.

    \item How to plot Root Locus for a given loop transfer function. That includes all the
    rules we have learned, e.g., the angles condition and the magnitude condition, finding the
    centroid and the angles of the asymptotes, finding the angles of arrivals to zeros and
    departures from poles, finding the crossing points with the jw axis using Routh-Hurwitz method, etc.

    We have the conditions on magnitude and angle.
    \[|K|\frac{|b(s_0)|}{|a(s_0)|}=1\qquad\sum_{i=1}^m\angle(s_0-z_i)-\sum_{j=1}^n\angle(s_0-p_j)=\begin{cases}
        (2l+1)\pi & K>0\\
        2l\pi & K<0
    \end{cases}\]
    If the number of poles and zeros on real axis to the right is odd, it is on the root locus, otherwise
    it is on the complementary root locus. With \(n\) poles and \(m\) zeros, \(n-m\) roots go to infinity.
    The asymptote centroid and angles are given by
    \[\sigma=\frac{\sum_{j=1}^np_j -\sum_{i=1}^m z_i}{n-m}\qquad\theta_r=\begin{cases}
        \frac{(2r+1)\pi}{n-m} & K>0\\
        \frac{2r\pi}{n-m} & K<0
    \end{cases}\]
    for \(r=\{0,\ldots, n-m-1\}\). Breakaway points of the characteristic equation \(1+KG(s)H(s)=0\) correspond
    to when there are multiple real roots at the same value. This occurs when
    \[\frac{d(G(s)H(s))}{s}=0\]
    has a roots that also satisfy the characteristic equation for a real \(K\). Use condition on angles to find angle
    of approach and departure from zeros and poles. Use Routh-Hurwitz method for intersection with imaginary axis.

    \item How to find out the desired closed-loop locations given the specifications on the
    transient response (e.g., settling time, overshoot percentage, etc.)

    \begin{gather*}
        \frac{\omega_n^2}{s^2 + 2\zeta\omega_n + \omega_n^2}\\
        \omega_n(-\zeta\pm\sqrt{\zeta^2-1})\\
        M_p=e^{-\frac{\pi\zeta}{\sqrt{1-\zeta^2}}}\\
        t_s=-\frac{1}{\zeta\omega_n}\ln(\beta\sqrt{1-\zeta^2})
    \end{gather*}

    \item What Lead, Lag, and Lead-Lag compensators are.

    The take the form of \(\frac{s-z_0}{s-p_0}\), and a Lead-Lag compensator is two chained together.
    Lead compensator has \(p_0<z_0<0\) while lag has \(z_0<p_0<0\).

    \item How to use Root Locus and design Lead compensators in order to place the closed-loop
    poles in desired locations.

    \item How to design Lag compensators in order to improve the steady-state tracking performance
    without disturbing the desired locations of the closed-loop poles.

\end{enumerate}

\begin{alignat*}{3}
    &f(t)&&\text{\quad}&&\mathcal{L}\left\{f(t)\right\}\\
    &1&&\text{\quad}&&\frac{1}{s}\text{,\quad}s>0\\
    &e^{at}&&\text{\quad}&&\frac{1}{s-a}\text{,\quad}s>a\\
    &t^n\text{,\quad}n=\text{positive integer}&&\text{\quad}&&\frac{n!}{s^{n+1}}\text{,\quad}s>0\\
    &t^p\text{,\quad}p>-1&&\text{\quad}&&\frac{\Gamma(p+1)}{s^{n+1}}\text{,\quad}s>0\\
    &\sin at&&\text{\quad}&&\frac{a}{s^2+a^2}\text{,\quad}s>0\\
    &\cos at&&\text{\quad}&&\frac{s}{s^2+a^2}\text{,\quad}s>0\\
    &\sinh at&&\text{\quad}&&\frac{a}{s^2-a^2}\text{,\quad}s>|a|\\
    &\cosh at&&\text{\quad}&&\frac{s}{s^2-a^2}\text{,\quad}s>|a|\\
    &e^{at} \sin bt&&\text{\quad}&&\frac{b}{(s-a)^2+b^2}\text{,\quad}s>a\\
    &e^{at} \cos bt&&\text{\quad}&&\frac{s-a}{(s-a)^2+b^2}\text{,\quad}s>a\\
    &t^n e^{at}\text{,\quad}n=\text{positive integer}&&\text{\quad}&&\frac{n!}{(s-a)^{n+1}}\\
    &u_c(t)&&\text{\quad}&&\frac{e^{-cs}}{s}\text{,\quad}s>0\\
    &u_c(t)f(t-c)&&\text{\quad}&&e^{-cs}F(s)\\
    &e^{ct}f(t)&&\text{\quad}&&F(s-c)\\
    &f(ct)&&\text{\quad}&&\frac{1}{c}F\left(\frac{s}{c}\right)\\
    &\int_0^t f(t-\tau)g(\tau)\,dt&&\text{\quad}&&F(s)G(s)\\
    &\delta(t-c)&&\text{\quad}&&e^{-cs}\\
    &f^{(n)}(t)&&\text{\quad}&&s^n F(s)-s^{n-1}f(0)-\ldots-f^{(n-1)}(0)\\
    &(-t)^n f(t)&&\text{\quad}&&F^{(n)}(s)
\end{alignat*}

\end{document}
